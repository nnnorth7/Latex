\chapter{Introduction to the Theory of Voting}

%%% section1
\section{Introduction}

In this chapter we mainly concentrate on \textit{multicandidate} voting with \textit{ranked} ballots, i.e. each voter submits a linear ordering of the alternatives, and single winners (or several winners, in the event of a tie) as outcomes. A voting rule in this setting is called a social choice function or $SCF$.\\

\noindent Here we introduce three most prominent results in multicandidate voting: 
\begin{itemize}
    \item \textbf{Majority cycles:} Collective peference may have a cycle, i.e. a majority of voters prefer some alternative a to b, a (different) majority prefers b to c, and a third majority prefers c to a;
    \item \textbf{Arrow's Theorem}: Every voting rule for three or more alternatives either violates IIA or is a dictatorship,in which the election outcome depends solely on the ballot of one designated voter;
    \item \textbf{Gibbard-Satterthwaite Theorem (GST)}: Every SCF $f$ other than a dictatorship fails to be strategyproof, i.e. $f$ sometimes provides an incentive for an individual voter i to manipulate the outcome, that is, to misrepresent his or her true preferences over the alternatives by casting an insincere ballot.\\
    In this case, the voter prefers the alternative that wins when she casts some insincere ballot to the winner that would result from a sincere one. A limitation of GST is that it presumes every election to have a unique winner, which might be problematic. Duggan-Schwartz Theorem provides a solution to it.
\end{itemize}

%%% section2
\section{Social Choice Functions: Plurality, Copeland, and Borda}

A multicandidate voting with ranked ballots is as follows:
\begin{itemize}
    \item $N = \{1,2, \dots, n\}$ is a finite set of voters.
    \item $A = \{1,2, \dots, m\}$ is a finite set of alternatives, with $m \geq 2$.
    \item A linear ordering $\succsim_i$ of $A$ stands for the ballot cast by voter $i$: $\succsim_i$ is transitive, complete and antisymmetric; $x \succ y$ is a shortwrite for $x \succsim y$ holds and $y \succsim x$ fails.
    \item $\linearor$ denotes the set of all linear orderings for $A$.
    \item A profile $\profile = (\succsim_1,\succsim_2,\dots,\succsim_n) \in \linearor^n $ is a collection of each voter's ballot.
    \item $\linearor^{<\infty}$ stands for $\bigcup_{n \in \N} \linearor^n$
\end{itemize}

Notice that $\linearor$ is actually a \textit{strict} ordering, no voter may express indifference to two alternatives. The ballots mentioned above are called \textit{preference rankings} since $x \succ_i y$ means voter $i$'s (strict) preference for alternative $x$ over alternative $y$.\\
Alternatively we might allow \textit{weak} preference rankings (\textit{pre-linear} orderings) as ballots. Thus let $R$ denote a profile of weak preference rankings and similarly to $\linearor$:
\begin{itemize}
    \item $\weakor$ denotes the set of all pre-linear orderings for $A$.
    \item $\weakor^n$ is the set of all profiles of weak rankings for a given $A$ and $n$.
\end{itemize}

In practice, we often use tabular to present a profile. For example, consider $\profile_1$ as following:
\begin{center}
    \begin{tabular}{cccc}
        102 & 101 & 100 & 1\\
        \hline
        $a$ & $b$ & $c$ & $c$\\
        $b$ & $c$ & $a$ & $b$\\
        $c$ & $a$ & $b$ & $a$
    \end{tabular}
\end{center}

$\profile_1$ is not a profile, but a \textit{voting situation}, i.e. a function $s:\ \linearor \to \N$. But mane voting rules are blind to the distinction between “profile” and “voting situation”, so we use the terms interchangeably.\\
~\\
Next we introduce some basic voting rules.

\subsection{Plurality}

A \textit{plurality ballot} names a single, most-preferred alternative, the \textit{plurality voting} rule selects the alternative(s) with a plurality (greatest number) of votes as the winner(s). \\
Under plurality voting, we only pay attention to the top-ranked alternatives and ignore the rest of the ranking. For example, in $\profile_1$, $a$ is the unique plurality winner, since there's 102 voters on $a$, even though it's not a majority.

\subsection{Copeland}

But it's unreasonable to ignore the rest part of the ranking. To make fuller use of the information in the ranking, we use \textit{net preference}:
~\\
\begin{definition}[net preference]
    $\mathit{Net}_\profile(a>b) = |\{j \in N\;|\; a \succ_j b\}| - |\{j \in N\;|\; b \succ_j a\}|$
\end{definition}

If the net preference for $a$ over $b$ is strictly positive ($\mathit{Net}_\profile(a>b) > 0$), then we say $a$ beats $b$ in the \textit{pairwise majority} sense, denoted as $a >^\mu b$ or $a >^\mu_\profile b$ with profile, where $>^\mu$ is the strict \textit{pairwise majority} relation, which is always complete for an odd number of voters with strict preferences; and $\geq^\mu$ is the weak version.

Also we can use tournament to depict $>^\mu$. Below is an example, where left is the pairwise majority tournament and right is weighted version for $\profile_1$.

\begin{center}
    \begin{tikzpicture}[->,>=stealth',shorten >=1pt,shorten <=1pt, auto,node distance=1.5cm,semithick]
        \node (a) at (0,0) {$a$};
        \node (b) at (1.5,2) {$b$};
        \node (c) at (3,0) {$c$};
        \node (a') at (5,0) {$a$};
        \node (b') at (6.5,2) {$b$};
        \node (c') at (8,0) {$c$};

        \path[->] (a) edge (b);
        \path[->] (b) edge (c);
        \path[->] (c) edge (a);
        \path[->] (a') edge node[left]{100} (b');
        \path[->] (b') edge node[right]{102} (c');
        \path[->] (c') edge node[below]{100} (a');
    \end{tikzpicture}
\end{center}

The next voting rule use \textit{Copeland scores} to select winners. A (symmetric) Copeland score of alternative $x$ is:
\begin{definition}[(symmetric) Copeland score]
    $\mathit{Copeland}(x) = |\{y \in A\;|\; x >^\mu y\}| - |\{y \in A\;|\; y >^\mu x\}|$
\end{definition}

The Copeland rule selects the alternative(s) with highest Copeland score. In $\profile_1$ for example, $a$'s Copeland scores is $0$, since $a \pma b$ and $c \pma a$. So as $b \mbox{ and } c$. Thus the winning set of $\profile_1$ is $\{a,b,c\}$.

\subsection{Borda}

In the case of Copeland rule, for every alternative $a$, all we care about is how many times $a$ has been defeated and won. However, as we can notice, the margins of victory or defeat matter as well. If we take these margins into consideration, things will be different.

Given a profile $\profile$, the \textit{symmetric Borda score} of an alternative $x$ is\footnote{Noted that the definition here is nonstandard, the standard version will come up later.}:
\begin{definition}[symmetric Borda score]
    \label{sbor}
    $\mathit{Borda}^{\mathit{sym}}_\profile (x) = \sum_{y \in A} \mathit{Net}_\profile (x > y)$
\end{definition}

Under Borda rule, alternative(s) with the highest Borda score is the winner(s). In $\profile_1$, the scores of $a,b,c$ are $0,2,-2$ respectively, thus the winner is $b$.

A more common asymmetric Borda score is defined via the \textit{vector of scoring weights} (score vector) $\mathbf{w}$:

\begin{definition}[asymmetric Borda score]
    For $|A| = m$. Let $\mathbf{w} = {m-1,m-2,m-3,\dots,0}$. $\mathit{Borda}^{\mathit{asym}}_\profile (x)$ is the sum of points awarded to x by all voters, where for voter $i$, if $a$ is top-ranked then $a$ get $(m-1)$ points, if the next then $(m-2)$\dots\dots, and $0$ to the least preferred.
\end{definition}

Noted that the two versions of Borda score are affinely equivalent, with $\mathit{Borda}^{\mathit{asym}}_\profile (x) = n + \frac{1}{2} \mathit{Borda}^{\mathit{sym}}_\profile (x)$, so they induce the same SCF\footnote{Don't understand what `affinely' mean\dots\dots}. Actually, if we make $\mathbf{w} = \{m-1,m-3,m-5,\dots,-(m-1)\}$, then we replicate the scores from \cref{sbor}. An adventage of symmetric approach is that it's well-defined for profiles which is weak preferences, thus the symmetric Copeland and Borda rules can be extended to weak preference cases.

\subsection{Social Choice Function}

Let $\mathcal{C}(X)$ denote the set of all nonempty subsets of a set $X$.

\begin{definition}[social choice function]
    \label{SCF}
    \begin{itemize}
        \item A \textit{social choice function} (SCF) is a map $f:\; \linearor^n \to \ca$ that returns a nonempty set of alternatives for each profile of strict preference.
        \item If $f(\profile) = 1$ then $f$ is \textit{single valued} on $\profile$ and sometimes we write $f(\profile) = x$ instead of $f(\profile) = \{x\}$.
        \item A \textit{resolute} SCF is one with no tied: it's single valued on all profiles.
    \end{itemize}
\end{definition}

Above are fixed electorate SCFs. We can substitude $\linearor^\infty$ for $\linearor^n$, from which we get variable electorate SCFs. Notice that the rest of this chapter presumes a fixed electorate, except where explicitly noted otherwise.

In \cref{SCF}, the value of the function is a group of winners, while in preceding three rules, we not only concentrate on the winner(s), but also the "social ranking"---one alternative is ranked over another if it has a higher score. Here we don't restrict the ranking to be strict, which leads to the definition below:

\begin{definition}[social welfare function]
    A \textit{social welfare function} (SWF), is a map $f:\; \linearor^n \to \weakor$ that returns a weak ranking of the set of alternatives for each profile of strict preferences.
\end{definition}

\subsection{Strategic Manipulation}

Now we take a look at some occasions where voters may have an incentive to cast insincere ballots under such voting rules.

Consider Ali, one of the two $a \succ b \succ c \succ d \succ e$ voters of profile $\profile_2$.
\begin{center}
    \begin{tabular}{ccc}
        2 & 3 & 2\\
        \hline
        $e$ & $ d $ & $ a$\\
        $c $&$ e $&$ b$\\
        $a $&$ b $&$ c$\\
        $d $&$ c $&$ d$\\
        $b $&$ a $&$ e$
    \end{tabular}
\end{center}

Under Copeland, Ali's least preferred alternative $e$ wins: $e$'s (symmetric) Copeland score is $2$, $b$'s is $-2$ and the other scores are each $0$. However, if Ali misrepresents his sincere preferences as a reverse of his ranking, the Copeland winner shifts to $d$, where $d \succ_{Ali} e$, with a score of $4$, the maximum possible.

\begin{definition}
    \begin{itemize}
        \item An SCF $f$ is \textit{single voter manipulable} if for some pair $\profile,\, \profile'$ of profiles on which $f$ is single valued, and voter $i$ with $\succsim'_j = \succsim_j$ for all $j \neq i$, $f(\profile') \succ_i f(\profile)$;
        \item $f$ is \textit{single voter strategyproof} if it's not single voter manipulable.
    \end{itemize}
\end{definition}

$\succ_i$ stands for the sincere preference, while $\succ'_i$ is the insincere one. An SCF being single voter manipulable means that if voters give an insincere ranking then he'll get a better result.

Under Copeland rule, Ali has to reverse the ballot. And under Borda rule, he can still do such things---by just lifting $d$ to the top position ($d \succ a \succ b \succ c \succ e$).

Plurality voting is not single voter manipulable, since it only care about the top-ranked alternatives. Any voter who wants $x$ rather than $y$ to be the winner won't place $y$ in the top on his sincere ballot, so he cannot lower $y$'s score. But if there's two voters switching there ballot, then Plurality voting can be manipulable as well.

Obviously voting rules with ties or single voter manipulable is inappropriate, thus we have to deal with such problem. That's what we do in the later section.