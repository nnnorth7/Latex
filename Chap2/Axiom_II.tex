\section{Axioms II: Reinforcement and Monotonicity Properties}

In this part, we mainly discuss the second group of the axioms, which are about profile changes when adding more voters or changing several voters' ballots. Hence we need variable electorate context and voting situation: for $s,t:\; \linearor \to \mathbb{Z}^+$\footnote{$s,t$ can be regarded as two group of voters with different ballots.}. For simplicity, we fix the following conventions:
\begin{itemize}
    \item $s+t$ stands for putting $s$ and $t$ together in one voter-set;
    \item $ks\;(k \in \mathbb{N})$ stands for replacing each individual voter of s with k “clones”.
\end{itemize}

\begin{definition}[Reinforcement (aka Consistency)]
    An SCF $f$ is reinforcing if $f(s) \cap f(t) \neq \emptyset \Rightarrow f(s+t) = f(s) \cap f(t)$.
\end{definition}

Intuitively, reinforcement requires that the common winning alternatives chosen by two disjoint sets of voters (if there exists) be exactly those chosen by the union of these sets. Specially if $f(s) = f(t)$ then $f(s+t) = f(s) = f(t)$.

There's also a weak form of reinforcement:
\begin{definition}[Homogeneity]
    $f(ks) = f(s)$ for all $k \in \N$.
\end{definition}

Noted that all scoring rules are reinforcing, since if some alternatives get the highest score both in $s$ and $t$, then they must have the highest score in $s+t$. Analogously we can apply the same argument to \textit{compound scoring rules}.

\begin{definition}[Compound Scoring Rules]
    A voting rule is a \textit{compound scoring rule} if any ties resulting from a first score vector $\mathbf{w}_1$ may be broken by score differences arising from a second such vector $\mathbf{w_2}$, with a possible third vector used to break ties that still remain, and so on; any finite number $j \geq 1$ of score vectors may be used.\\
    A \textit{simple scoring rule} is the scoring rule which is not compound.
\end{definition}

Notice that on a domain that is restricted by fixing an upper bound on the number of voters, every such compound rule is equivalent to some simple scoring rule\footnote{Why??????Anyone can tell me why\dots?????????}.

\begin{theorem}[See \textcite{John_compound} and \textcite{young_social_1975}]
    The anonymous, neutral, and reinforcing SCFs are exactly the compound scoring rules.
\end{theorem}

\begin{proof}
    886
\end{proof}

\begin{proposition}
    \label{not_reinforcing}
    All Condorcet extension SCFs for three or more alternatives violate reinforcement.
\end{proposition}

\begin{proof}
    Consider voting situation $s$ and $t$ with 3 alternatives below:\\
    ~\\
    \begin{minipage}{0.5\textwidth}
        \begin{center}
            \begin{tabular}{ccc}
                2 & 2 & 2\\
                \hline
                a & c & b\\
                b & a & c\\
                c & b & a
            \end{tabular}
        \end{center}
    \end{minipage}
    \begin{minipage}{0.4\textwidth}
            \begin{tabular}{cc}
                2 & 1\\
                \hline
                b & a \\
                a & b\\
                c & c
            \end{tabular}
    \end{minipage}
    \\
    Let $f$ be any Condorcet extension. Since $b$ is the Condorcet winner in $t$, we have $f(t) = \{b\}$. Also $a$ is the Condorcet winner in $(s+t)$, thus $f(s+t) = \{a\}$. If $f$ is reinforcing then $a \in f(t)$ should hold, which leads to a contradiction.\\
    If we assume $f$ to be Pareto, then it's easy to extend our construction to a general form, just by adding other alternatives $x_i$ behind $a,b,c$ in the voting situation. If not, we have to find another more complicated voting situation as a counterexample.
\end{proof}

\noindent Intuitively, for a winner $a$, if we move $a$ from under some alternatives to over them in some voters' preference, without changing the relative order of other pair of alternatives excluded $a$, then $a$ still should be the winner (\textit{simple lift}). That's what \textit{monotonicity} said.

\begin{definition}
    A resolute SCF $f$ satisfies \textit{monotonicity} (aka weak monotonicity) if whenever a profile $\profile$ is modified to $\profile'$ by having one voter $i$ switch $\succsim_i$ to $\succsim'_i$ by lifting the winning alternative $x = f(\profile)$ simply, $f(\profile') = f(\profile)$.
\end{definition}

Notice that here we require that $f(\profile)$ is single-valued, if not, then the definition only cares about the cases when $|f(\profile)| = 1$.

Suppose $f$ is a voting rule which select the alternative(s) with highest score, and lifting alternatives $x$ never lower $x$'s score or raise $y$'s score for $ y \neq x$. Then $f$ is always monotonic. It follows that all proper scoring rules are monotonic.
However, each of scoring run-off rule is not monotonic\footnote{For scoring run-off rules, see the last part of \cref{run-off}.}. I omitted the proof here since it seems to do with some knowledge of computation\dots

\begin{remark}
    Every resolute SCF $f$ which violate monotonicity implies that there's an opportunity for voter $i$ manipulate $f$ by simple lifting or dropping an alternative:\\
    Let $\succ_i \mapsto \succ'_i$ by a simple lift of he winning alternative $a$ which makes $b$ win and $a$ lose. If $b \succ_i a$, then the voter with sincere ballot $\succ_i$ would gain by casting the insincere ballot $\succ'_i$ and vice versa.
\end{remark}

Thus monotonicity is a weak form of strategyproofness. Below are some relavent definitions:

\begin{definition}
    A resolute SCF $f$ satisfies:
    \begin{itemize}
        \item \textit{Strategyproofness} if whenever a profile $\profile$ is modified to $\profile'$ by having one voter $i$ switch $\succsim_i$ to $\succsim'_i$ , $f(\profile) \succsim_i f(\profile')$.
        \item \textit{Maskin monotonicity} (aka \textit{strong monotonicity}) if whenever a profile $\profile$ is modified to $\profile'$ by having one voter $i$ switch $\succsim_i$ to a ballot $\succsim'_i$ satisfying for all $y,\; f(\profile) \succsim_i y \Rightarrow f(\profile) \succsim'_i y$ then $f(\profile') = f(\profile)$\footnote{See \textcite{maskin1999}. The original definition is based on not only resolute SCF: $f$ is Maskin monotonic if $\forall a \in f(\profile)$, if $\forall i \in N,\; \forall b \in A,\; a \succsim_i b \Rightarrow a \succsim'_i b$, then $a \in f(\profile')$. It can be understood as, if $a$ does not fall below any alternatives that it was not below before, then $a$ will still be the winner.}.
        \item \textit{Down monotonicity} if whenever a profile $\profile$ is modified to $\profile'$ by having one voter $i$ switch $\succsim_i$ to $\succsim'_i$ by dropping a losing alternative $b \neq f(\profile)$ simply, $f(\profile') = f(\profile)$.
        \item \textit{One-way monotonicity} if whenever a profile $\profile$ is modified to $\profile'$ by having one voter $i$ switch $\succsim_i$ to $\succsim'_i$, $f(\profile) \succsim_i f(\profile')$ or $f(\profile') \succsim_i f(\profile)$\footnote{See \textcite{sanver_one-way_2009}: It asserts that whenever one identification represents a successful manipulation, the other represents a failure.}
        \item \textit{Half-way monotonicity} if whenever a profile $\profile$ is modified to $\profile'$ by having one voter $i$ switch $\succsim_i$ to $\succsim^{rev}_i$, $f(\profile) \succsim_i f(\profile')$, where $\succsim^{rev}$ denotes the reverse of $\succsim$.
        \item \textit{Participation} (the absence of no show paradoxes) if whenever a profile $\profile$ is modified to $\profile'$ by adding one voter $i$ with ballot $\succsim_i$ to the electorate, $f(\profile') \succsim_i f(\profile)$.
    \end{itemize}
\end{definition}

Participation is a corresponding version of strategyproofness for no-show paradox. No-show paradox, first come up by \textcite{fishburn_paradoxes_1983}, also mentioned by \textcite{sanver_one-way_2009}, shows that, one additional participating voter shows up to cast her vote, and the winner is then an alternative that is strictly inferior (according to the preferences of the participating voter) to the alternative who would have won had she not shown up. Thus the paradox implies an opportunity to manipulate by abstaining.

\begin{proposition}
    For resolute SCFs:
    \begin{enumerate}
        \item Strategyproofness $\Rightarrow$ Maskin monotonicity $\Leftrightarrow$ Down monotonicity $\Rightarrow$ monotonicity
        \item Strategyproofness $\Rightarrow$ One-way monotonicity $\Rightarrow$ Half-way monotonicity
        \item Participation $\Rightarrow$ monotonicity
    \end{enumerate}
\end{proposition}

\begin{proof}
    The proof of \textit{Item I} and \textit{Item II} is straightforward, except for the first arrow in Item I. Consider an SCF $f$ which violates Maskin monotonicity. Let $a, b \in A$, $\profile$ be a profile where $f(\profile) = \{a\}$ and for voter $i$, $b \succsim_i a$. Now $\profile'$ is modified from $\profile$ by changing $b \succsim_i a$ to $a \succsim_i b$, and $f(\profile') \neq \{a\}$. 
    If $f(\profile') = \{b\}$ or any other alternatives $c$ with $c \succsim_i a$, then a voter with sincere preference $\succsim_i$ would gain a better result by casting the insincere ballot $\succsim'_i$. Otherwise a voter with sincere preference $\succsim'_i$ can change his ballot to $\succsim$. Thus $f$ is not strategyproofness.\\
    For \textit{Item III}, see \textcite{sanver_one-way_2009}. Note that the original proof aims at proving participation $\Rightarrow$ one-way monotonicity,  but failed.\\
    Let $\profile \in \mathrm{Dom}(f)$ be a profile and $s,t$ be two preference given by two voters $s \mbox{ and } t$ seperately. Assume that $f$ is an SCF satisfying participation with $f(\profile \wedge s) = a$ and $f(\profile \wedge t) = b$, we show that $a \succsim_s b$ or $b \succsim_t a$. Consider 3 cases:
    \begin{itemize}
        \item \textit{Case 1:} $f(\profile) = a$ or $f(\profile) = b$. If $f(\profile) = a$, since $f(\profile \wedge t) = b$, $b \succsim_t a$ by participation; if $f(\profile) = b$, then similarly $a \succsim_s b$.
        \item \textit{Case 2:} $f(\profile \wedge s \wedge t) = a$ or $f(\profile \wedge s \wedge t) = b$. It's analogous to the preceding case.
        \item \textit{Case 3:} $f(\profile) = x$ with $x \not \in \{s,t\}$ and $f(\profile \wedge s \wedge t) = y$ with $y \not \in \{s,t\}$. By participation we have $a \succsim_s x$, $b \succsim_t x$, $y \succsim_s b$ and $y \succsim_t a$. If $\succsim_s = \succsim^{rev}_t$ then $a \succsim_s x$ and $x \succsim_s b$, thus $a \succsim_s b$ as desired.
    \end{itemize}
\end{proof}

Note that the definition of participation here is a typical form for fixed-electorate SCF, while as a property of variable-electorate SCFs, participation cannot follow from strategyproofness.

\begin{theorem}
    \label{conext}
    Let $f$ be any resolute Condorcet extension for four or more alternatives. Then
    \begin{enumerate}
        \item $f$ violates participation (if $f$ is a variable-electorate SCF) and
        \item $f$ violates half-way monotonicity (if $f$ is a fixed-electorate SCF for sufficiently large $n$).
    \end{enumerate}
\end{theorem}

\begin{proof}

\end{proof}

\begin{corollary}
    Let $f$ be a resolute SCF for four or more alternatives and sufficiently large odd $n$. If $f$ is neutral and anonymous on $\condom$, then either $f$ fails to be strategyproof on $\condom$,or $f$ violates half-way monotonicity.
\end{corollary}

\begin{proof}
    Recall \cref{ThCK}, we know that PMR is the unique resolute, anonymous, neutral and strategyproof SCF on $\condom$ for odd $n$. Thus for $f$ satisfying the conditions above, if $f$ is a strategyproof on $\condom$, then $f$ must be a Condorcet extension. By \cref{conext}, $f$ violates half-way monotonicity.
\end{proof}


\section{Voting Rules II: Kemeny and Dodgson}

John Kemeny defined a neutral, anonymous, and reinforcing Condorcet extension that escapes the limitation in \cref{not_reinforcing} by using the \textit{social preference function}, whose outcome is a set of one or more rankings. Here is his basic method.

\begin{definition}
    \begin{itemize}
        \item The \emph{Kendall tau metric} $d_K$ measures the distance between two linear orderings $\succ,\;\succ^\star$ by counting pairs of alternatives on which they disagree, i.e. 
        $$d_K (\succ,\;\succ^\star) = |\{(a,b) \in A^2 \;|\; a \succ b \mbox{ and } b \succ^\star a\}|.$$
        \item $d$ on profiles is an extension of $d$ on a ballot, where
        $$d(\profile,\profile') = \sum_{i = 1}^{n} d (\succ, \succ')$$
        \item For each $\succ$ define the unanimous profile $U^\succ$ by $U^\succ_i = \succ$ for all i.
        \item For any profile $\profile$, the \emph{Kemeny Rule} returns the ranking(s) $\succ$ minimizing $d_K(\profile,U^\succ)$.
    \end{itemize}
\end{definition}

$d_K(\succ,\succ^\star)$ gives the minimum number of sequential inversions needed to convert $\succ$ to $\succ^\star$ and analogously we get what $d_K (\profile,U^\succ)$ means. Noted that Kemeny Rule is a Condorcet extension: If $a$ is $\profile$'s Condorcet winner and $\succ$ does not put $a$ on top, then lifting $a$ simply to the top would strictly decrease $d_K(\profile,U^\succ)$, thus $d_K(\profile,U^\succ)$ is not minimal and all rankings in the Kemeny outcome place $a$ on top, from which we have Kemeny Rule is a Condorcet extension.

Kemeny Rule is also a scoring rule, although not in the sense of \cref{scoring_rule}. Consider a definition in the following:

\begin{definition}[ranking score rule]
    \begin{itemize}
        \item A \emph{ranking score function} $\scoringvec:\; \linearor \times \linearor \to \mathbb{R}$ assigns a real number scoring weight $\scoringvec(\succ^\star, \succ)$ to each pair of rankings.
        \item Any such function induces a \emph{ranking scoring rule}, in which a voter with ranking $\succ_i$ awards $\scoringvec(\succ_i, \succ)$ points to each ranking $\succ \in \linearor$. All points awarded to a given ranking are summed, and the winner is the ranking(s) with greatest sum.
    \end{itemize}
\end{definition}

The definition above extends the range of scoring rule to include Kemeny Rule. The scoring weight $\scoringvec(\succ^\star, \succ) = \frac{m(m-1)}{2} - d_K(\succ^\star,\succ)$.

It's obvious that Kemeny Rule is neutral and anonymous. For Reinforcement, since the outcome is no longer a winning set, but a set of rankings, the reinforcement requirement have actually been weakened, where $f(s+t)$ denote \emph{sets of rankings}. Consider the requirement of reinforcement:
$$
f(s) \cap f(t) \neq \emptyset \Rightarrow f(s+t) = f(s) \cap f(t)
$$
but for the situation likely to \cref{not_reinforcing}, $f_{\textit{Kem}}(s) = \{a \succ b \succ c,\; b \succ c \succ a,\; c \succ a \succ b\}$, $f_{\textit{Kem}} = \{b \succ a \succ c\}$ and $f(s) \cap f(t)$ is empty.\\
~\\
\begin{theorem}[\parencite{Young_1978}]
    Among social preference functions Kemeny's rule is the unique neutral and reinforcing Condorcet extension.
\end{theorem}

Another famous voting rule which is often compared to Kemeny's rule is proposed by Charles Dodgson.

\begin{definition}
    For any profile $\profile$ the Dodgson rule returns the Condorcet winner(s) for the profile(s) $\profile' \in \condom$ minimizing $d_K (\profile,\profile')$ among all $P' \in \condom$.
\end{definition}

Next paragraph is cited from \textcite{brandt_handbook_2016}:
``Both Kemeny and Dodgson may be interpreted as minimizing a distance to “consensus.” They use the same metric on rankings, but different notions of consensus: unanimity for Kemeny versus membership in $\condom$ for Dodgson. It is not difficult to see that every preference function that can be defined by minimizing distance to unanimity is a ranking scoring rule,75 hence is reinforcing in the preference function sense. We can convert a preference function into an SCF by selecting all top-ranked alternatives from winning rankings, but this may transform a reinforcing preference function into an nonreinforcing SCF—as happens for Kemeny. The conversion preserves homogeneity, however, so every distance-from-unanimity minimizer is homogeneous as a social choice function. In this light, the inhomogeneity of Dodgson argues an advantage for unanimity over $\condom$ as a consensus notion.''