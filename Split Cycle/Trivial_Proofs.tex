\chapter{``Trivial'' Proofs}

\begin{lemma}
    Let $F$ be a voting method. Then (i) if $F$ is Condorcet consistent, $\profile \in dom(F)$ with $|X(\profile)| > 1$, and $c$ is a Condorcet winner in $\profile$, then $c$ is the unique Condorcetian candidate for $F$ in $\profile$; and (ii) if for any $\profile \in dom(F)$ with a unique Condorcetian candidate $c$, $F(\profile) = \{c\}$, then $F$ is Condorcet consistent.
\end{lemma}

\begin{proof}
    Proof for part (i) is omitted.\\
    For part (ii), suppose $c$ is a Condorcet winner in $\profile$, and $F$ selects the unique Condorcetian candidate as winner if there is a such candidate. We use induction on the number of $X(\profile)$ to prove it. \\
    If $|X(\profile)| = 1$, obviously $F(\profile) = \{c\}$, hence $F$ is Condorcet consistent for profiles with only one candidate. \\
    Now consider the case when $|X(\profile)| = n$ and $n > 1$. By induction hypothesis, $F$ is Condorcet consistent for any profiles $\profile'$ with $X(\profile') < n$. We show that $c$ is the unique Condorcetian candidate, then by supposition $F(\profile) = \{c\}$ and $F$ is Condorcet consistent.\\
    Since $c$ is a Condorcet winner in $\profile$, $\Margin_\profile(c,b) > 0$ holds for all $b \in X(\profile) \backslash \{c\}$ and hence $c$ is a Condorcet winner in $\profile_{-b}$ for some $b \in X(\profile) \backslash \{c\}$. By induction hypothesis, $c \in F(\profile_{-b})$, thus $c$ is a Condorcetian candidate. For uniqueness, we claim that for all $b \in X(\profile)$ and $b \neq c$, $b$ is not a Condorcetian candidate, since for all $\profile_{-a}$ with $a \in X(\profile)$ and $\Margin_\profile (b,a) >0$, $c$ is the Condorcet winner and $b \not \in F(\profile_{-a}) = \{c\}$.
\end{proof}

Note that in the paper, Wesley defines $a$ as a Condorcet winner if `$a$ wins against every other candidate head-to-head' and there might be a misunderstanding of `wins against'\footnote{Solved. Definition of Condorcet winner is given in Section 5. Doesn't he think it is a little bit weird? Yifeng Ding and I keep looking for the definition for nearly 2 minites!:$\cdot$(}. We take the definition here as a strongly Condorcet winner. which says if $a$ is a Condorcet winner, for any $b \in X(\profile)$, $\Margin(a,b) > 0$. In fact, if we use the weak form ($\Margin(a,b) \geq 0$), we may come into trouble. Consider the following case\footnote{For simplicity, we omit margins here.}:

\begin{center}
    \begin{tikzpicture}
        \node[candidate] (a) [] {{$a$}};
        \node[candidate] (b) [right=of a] {{$b$}};
        \node[candidate] (c) [below=of a] {{$c$}};
        \node[candidate] (d) [right=of c] {{$d$}};

        \path[->] (a) edge (b);
        \path[->] (a) edge (c);
        \path[->] (b) edge (c);
        \path[->] (c) edge (d);
        \path[->] (d) edge (b);
    \end{tikzpicture}
\end{center}

In the case above, $a$ is the Condorcet winner in $\profile$. But $a,d$ are both Condorcet winners in $\profile_{-c}$. Since $F(\profile_{-c}) \subseteq \{a,c\}$, we are not sure whether $a \in F(\profile_{-c})$. Applying such situation to another profile may cause $a$ not being a Condorcetian candidate.\footnote{But as we can see in above case, $a$ is still a Condorcetian candidate since $a \in F(\profile_{-b})$, and what Yifeng Ding want to show is $b$ is not Condorcetian (I guess), so maybe there's no such question\dots}