\chapter{Definition and Properties}

Here I'll put a collection of basic definitons and properties required in the article. In the next chapter, I'll complete the details of some specific proofs in the article.

\section{Definitions}

\begin{definition}[Basic definition]
    \begin{itemize}
        \item $\mathcal{V}$ and $\mathcal{X}$ are infinite sets of \emph{voters} and \emph{candidates}. $V$ and $X$ are finite subsets of them respectively.
        \item $\mathcal{B}(X)$ is the set of all asymmetric binary relations on $X$.
        \item A \emph{profile} is a pair $(\profile, X(\profile))$ where $\profile:\; V(\profile) \to \mathcal{B}(X(\profile))$ for some nonempty finite $X(\profile) \subseteq \mathcal{X}$ and nonempty finite $V (\profile) \subseteq \mathcal{V}$. We conflate the profile with the function $\profile$. $X(\profile)$ and $V (\profile)$ the sets of candidates in $\profile$ and voters in $\profile$, respectively. We call $\profile(i)$ voter $i$'s ballot, and we write `$x\profile_iy$' for $(x, y) \in \profile(i)$.
        \item Different types of \emph{profiles}:
        \begin{enumerate}
            \item $\mathscr{P}$: the class of all profiles;
            \item $\mathscr{A}$: the class of \emph{acyclic profiles}, in which each voter's ballot is \emph{acyclic}, meaning that there are no $x_1, \dots, x_n \in X(\profile)$ with $n > 1$ s.t. for $k \in \{1,\dots,n-1\}$, we have $x_k \profile_i x_{k+1}$ and $x_n = x_1$;
            \item $\mathscr{I}$: the class of \emph{strict weak order profiles}, which means every voter's ballot is asymmetric and negatively transitive;
            \item $\mathscr{L}$: the class of \emph{linear profiles} (transitive and total).
        \end{enumerate}
        \item A \emph{margin graph} is a weighted directed graph $\mathcal{M}$ with positive integer weights whose edge relation is asymmetric. We say $\mathcal{M}$ has \emph{uniform parity} if all weights of edges are even or all weights of edges are odd, and if there are two nodes with no edge between them, then all weights are even.
        \item Let $\profile$ be a profile and $a,b \in X(\profile)$. Then 
        \[\Margin_\profile (a,b) = |\{i \in V(\profile)\;|\; a \profile_i b\}| - |\{i \in V(\profile)\;|\; b \profile_i a\}|\]
        \item Given a set $\mathscr{D}$ of profiles, a voting method on $\mathscr{D}$ is a function $F$ such that for all profiles $\profile \in \mathscr{D}$, we have $\emptyset \neq F(\profile) \subseteq X(\profile)$. We call $F(\profile)$ the set of winners or winning set for $\profile$ under $F$ . We write $dom(F )$ for the set $\mathscr{D}$ on which $F$ is defined.
    \end{itemize}
\end{definition}

\begin{remark}
    Here we take asymmetry as a foundational property for ballot. But others may regard reflexivity as basis, see \textcite{Heitzig2002}.
\end{remark}

\begin{definition}[Split Cycle]
    \begin{itemize}
        \item A cycle is a \emph{simple cycle} if for all distinct $i,j \in \{1,\dots,n\}$, $x_i = x_j$ only if $i,j \in \{1,n\}$ (i.e. all nodes are distinct except $x_1 = x_n$).
        \item Let $\profile$ be a profile and $\rho$ a simple cycle in $\mathcal{M}(\profile)$. The splitting number of $\rho$, $\splitnum_\profile(\rho)$, is the smallest margin between consecutive candidates in $\rho$.
        \item Let $\profile$ be a profile and $a, b \in X(\profile)$. The \emph{cycle number} of $a$ and $b$ ($\cyclenum_\profile (a,b)$) in $\profile$ is
        \[\mbox{max}(\{0\} \cup \{\splitnum(\rho)\;|\;\rho \mbox{ a simple cycle } a \to b \to \dots \to a \}).\]
        \item Let $\profile$ be a profile and $a,b \in X(\profile)$. Then $a$ \emph{defeats} $b$ in $\profile$ according to Split Cycle if $\Margin_\profile(a,b) > 0$ and 
        \[\Margin_\profile(a,b) > \splitnum(\rho)\ \mbox{for every simple cycle}\ \rho\ \mbox{in}\ \mathcal{M}(\profile) \mbox{ containing } a \mbox{ and } b\]
        or
        \[\Margin_\profile(a,b) > \cyclenum_\profile (a,b).\]
        \item A candidate $b$ is undefeated in $\profile$ if there is no candidate who defeats $b$ and for any profile $\profile$, the set of Split Cycle winners, $SC(\profile)$, is the set of candidates who are undefeated in $\profile$.
    \end{itemize}
\end{definition}

Discussions about \emph{functional collective choice rule} (mainly in Remark 4.25) was omitted, since I didn't understand what he actually said\dots

\section{Properties}

\begin{definition}
    \begin{itemize}
        \item A voting method $F$ is \emph{quasi-resolute} if for every uniquely-weighted $\profile \in dom(F)$, $|F (\profile)| = 1$.
        \item Given a voting method $F$, profile $\profile$, and $a \in X(\profile)$, we say that $a$ is \emph{Condorcetian} for $F$ in $\profile$ if there is some $b \in X(\profile)$ such that $a \in F(\profile_{-b})$ and $\Margin_\profile(a, b) > 0$ (for \emph{weakly Condorcetian}, $\Margin_\profile(a, b) \geq 0$).
        \item A voting method $F$ satisfies \emph{expansion consistency} if for all $\profile \in dom(F)$ and nonempty $Y, Z \subseteq X(\profile)$ with $Y \cup Z = X(\profile)$, we have $F(\profile|_Y ) \cap F (\profile|_Z ) \subseteq F (\profile)$.
    \end{itemize}
\end{definition}

\begin{remark}
    Note that expansion consistency is different from reinforcement.
\end{remark}

\begin{definition}[Spoil and Steal]
    \begin{itemize}
        \item \textbf{Spoiling and stealing}:
        \begin{enumerate}
            \item $b$ \emph{spoils the election for} $a$ in $\profile$ if $a \in F (\profile_{-b})$, $\Margin_\profile(a, b) > 0$, $a \not \in  F (\profile)$, and $b \not \in F(\profile)$;
            \item $b$ \emph{steals the election from} $a$ in $\profile$ if $a \in F (\profile_{-b})$, $\Margin_\profile(a, b) > 0$, $a \not \in  F (\profile)$, and $b \in F(\profile)$;
        \end{enumerate}
        \item \textbf{Immunity}: Let $F$ be a voting method.
        \begin{enumerate}
            \item $F$ satisfies \emph{immunity to spoilers} if for $\profile \in dom(F)$ and $a, b \in X(\profile)$, $b$ does not spoil the election for $a$.
            \item $F$ satisfies \emph{immunity to stealers} if for $\profile \in dom(F)$ and $a, b \in X(\profile)$, $b$ does not steal the election from $a$.
            \item $F$ satisfies \emph{stability for winners} if for $\profile \in dom(F)$ and $a, b \in X(\profile)$, if $a \in F (\profile_{-b})$ and $\Margin_\profile(a, b) > 0$, then $a \in F(\profile)$.
        \end{enumerate}
        \item \textbf{Partial immunity}: Let $F$ be a voting method. 
        \begin{enumerate}
            \item $F$ satisfies \emph{partial immunity to spoilers} if for all $\profile \in dom(F)$ and $a, b \in X(\profile)$, if $a$ is the unique Condorcetian candidate in $\profile$, then $b$ does not spoil the election for $a$.
            \item $F$ satisfies \emph{partial immunity to stealers} if for all $\profile \in dom(F)$ and $a, b \in X(\profile)$, if $a$ is the unique Condorcetian candidate in $\profile$, then $b$ does not steal the election from $a$.
            \item $F$ satisfies \emph{partial stability for winners} if for all $\profile \in dom(F)$ and $a \in X(\profile)$, if $a$ is the unique Condorcetian candidate in $\profile$, then $a \in F(\profile)$.
            \item A voting method $F$ satisfies \emph{stability for winners with tiebreaking} if for all $\profile \in dom(F)$, if some candidate is Condorcetian for $F$ in $\profile$, then all candidates in $F (\profile)$ are Condorcetian for $F$ in $\profile$.
        \end{enumerate}
        \item \textbf{Strong stability for winners:}
        \begin{enumerate}
            \item A voting method $F$ satisfies \emph{strong stability for winners} if for all $\profile \in dom(F)$, all candidates who are weakly Condorcetian for $F$ in $\profile$ belong to $F(\profile)$.
            \item A voting method $F$ satisfies \emph{strong stability for winners with tiebreaking} if not only does $F$ satisfy stability for winners with tiebreaking but also for all $\profile \in dom(F)$, if some candidate is weakly Condorcetian for $F$ in $\profile$ but no candidate is Condorcetian for $F$ in $\profile$, then all candidates who win in $\profile$ are weakly Condorcetian for $F$ in $\profile$.
        \end{enumerate}
    \end{itemize}
\end{definition}
~\\
Criteria in Section 5 are not present here since it's easy to read. Instead, I will make a simple clarification of voting method mentioned in Appendix C. Notice that I will only make a one-sentence explanation for each of them.

\section{Other Methods}

\begin{itemize}
    \item \emph{Ranked Pairs:} List all the edges according to margins from highest to lowest; delete all the edges in the margin graph; add the edges listed one by one; if adding a certain edge will cause a cycle, then skip the edge.
    \item \emph{Beat Path:} (i) $a$ defeat $b$ if $\textit{Strength}(a,b) \geq \textit{Strength}(b,a)$; (ii) $x$ is a winner if nobody defeat him.
    \item \emph{Minimax:} $x$ is Minimax winner if $x$'s largest majority loss is the smallest.
    \item \emph{GETCHA:} (Given by Wesley.) Let $\profile$ be a profile and $S \subseteq X(\profile)$. Then $S$ is $\to_\profile$-dominant if $S \neq \emptyset$ and for all $x \in S$ and $y \in X(\profile) \backslash S$, we have $x \to y$. Define
    \[\textit{GETCHA}(\profile) = \bigcap \{S \subseteq X(\profile)\;|\; S \mbox{ is }\to_\profile \mbox{-dominant}\}.\]
    \item \emph{GOCHA:} (Also by Wesley.) Let $\profile$ be a profile and $S \subseteq X(\profile)$. Then $S$ is $\to_\profile$-undominated if for all $x \in S$ and $y \in X(\profile) \backslash S$, we have $y \not \to_\profile x$. Define
    \[\textit{GOCHA}(\profile) = \bigcup\{S \subseteq X(\profile)\;|\; S \mbox{ is } \to_\profile \mbox{-undominated and no } S' \subseteq S \mbox{ is } \to_\profile \mbox{-undominated}\}\]
    \item \emph{Uncovered Set:} Two versions from Fishburn and Gillies. We say $y$ \emph{left-covers} $x$ in $\mathcal{M}$ if for all nodes $z$ in $\mathcal{M}$, if $z \to y$, then $z \to x$.
    \begin{align*}
        \textit{UC}_\textit{Fish} (\profile) = &\  \{x \in X(\profile)\;|\; \mbox{there is no } y \in X(\profile):\; y \mbox{ left-covers }x \mbox{ but } x \mbox{ does not left-cover } y\}; \\
        \textit{UC}_\textit{Gill} (\profile) = &\  \{x \in X(\profile)\;|\; \mbox{there is no } y \in X(\profile):\; y \to x \mbox{ and } y \mbox{ left-covers } x\}.
    \end{align*}
    \item \emph{Instant Runoff:} Remove the candidate with the least number of first-place votes, until there is a candidate with a majority of the first-place votes.
\end{itemize}